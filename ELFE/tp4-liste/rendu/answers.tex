\documentclass[a4paper,10pt]{article}
\include{amsymb}
\include{marvosym}

\usepackage[francais]{babel}
\usepackage[cyr]{aeguill}
\usepackage[applemac]{inputenc}
\usepackage{graphicx}
\usepackage{xspace}
\usepackage[a4paper]{geometry}
\usepackage{latexsym,amsmath,amssymb,textcomp}

\usepackage{pdfpages}

\usepackage{hyperref}


\newlength{\indentationnota}
\newlength{\largeurlignenota}
\newlength{\paddingnota}
\newlength{\largeurnota}
\setlength{\paddingnota}{5pt}
\setlength{\largeurnota}{0.9cm}


\makeatletter

\newenvironment{pictonote}[1]{% on passe le nom du �chier en argument
\begin{list}{}{%
\setlength{\labelsep}{0pt}%
\setlength{\rightmargin}{15pt}%
\setlength{\paddingnota}{5pt}}
\item%
\setlength{\indentationnota}{%
\@totalleftmargin+\largeurnota+\paddingnota}%
\setlength{\largeurlignenota}{%
\linewidth-\largeurnota-\paddingnota}%
\parshape=3%
\indentationnota\largeurlignenota%
\indentationnota\largeurlignenota%
\@totalleftmargin\linewidth%
\raisebox{-\largeurnota+2.2ex}[0pt][0pt]{%
\makebox[0pt][r]{%
\includegraphics[width=\largeurnota]{#1}%
\hspace{\paddingnota}}}%
\ignorespaces}{%
\end{list}}

\makeatother

\newenvironment{attention}%
{\begin{pictonote}{/Users/benjamin/Documents/Education/danger}}%
{\end{pictonote}}

\newcommand\danger{\raisebox{-0.4ex}{\LARGE $\triangle$ \normalsize} \hspace{-4.1ex}! \hspace{1ex}}
\newcommand\PL{Programmation Logique\xspace}
\newcommand\pl{\bsc{Prolog}\xspace}

\title{ELFE TP3: R�solution - R�cursivit�}
\author{Fran�ois \bsc{Lepan}\\Benjamin \bsc{Van Ryseghem}}

\begin{document}
\maketitle

\part{G�n�alogie}
\subsection*{Question 1.2}
22 �tapes sont franchies pour obtenir un r�sultat lors de l'ex�cution du but \verb?descendant_de(anne,emily).?

\subsection*{Question 1.3}
Nous obtenons 10 solutions:
\begin{enumerate}
	\item\vspace{1ex} \begin{minipage}{10cm}{	\begin{verbatim}
X = anne
Y = bridget  
						\end{verbatim}}
		\end{minipage}
	\item \vspace{2ex} \begin{minipage}{10cm}{	\begin{verbatim}
X = bridget
Y = caroline  
						\end{verbatim}}
		\end{minipage}
	\item \vspace{2ex} \begin{minipage}{10cm}{	\begin{verbatim}
X = caroline
Y = donna  
						\end{verbatim}}
		\end{minipage}
	\item \vspace{2ex} \begin{minipage}{10cm}{	\begin{verbatim}
X = donna
Y = emily  
						\end{verbatim}}
		\end{minipage}
	\item \vspace{2ex} \begin{minipage}{10cm}{	\begin{verbatim}
X = anne
Y = caroline  
						\end{verbatim}}
		\end{minipage}
	\item \vspace{2ex} \begin{minipage}{10cm}{	\begin{verbatim}
X = anne
Y = donna  
						\end{verbatim}}
		\end{minipage}
	\item \vspace{2ex} \begin{minipage}{10cm}{	\begin{verbatim}
X = anne
Y = emily  
						\end{verbatim}}
		\end{minipage}
	\item \vspace{2ex} \begin{minipage}{10cm}{	\begin{verbatim}
X = bridget
Y = donna  
						\end{verbatim}}
		\end{minipage}
	\item \vspace{2ex} \begin{minipage}{10cm}{	\begin{verbatim}
X = bridget
Y = emily  
						\end{verbatim}}
		\end{minipage}
	\item \vspace{2ex} \begin{minipage}{10cm}{	\begin{verbatim}
X = caroline
Y = emily  
						\end{verbatim}}
		\end{minipage}
\end{enumerate}

\subsection*{Question 2.2}
Il y a une boucle infinie.

\subsection*{Question 2.3}
Nous obtenons 10 solutions:
\begin{enumerate}
	\item\vspace{1ex} \begin{minipage}{10cm}{	\begin{verbatim}
X = anne
Y = emily 
						\end{verbatim}}
		\end{minipage}
	\item \vspace{2ex} \begin{minipage}{10cm}{	\begin{verbatim}
X = anne
Y = donna
						\end{verbatim}}
		\end{minipage}
	\item \vspace{2ex} \begin{minipage}{10cm}{	\begin{verbatim}
X = anne
Y = caroline
						\end{verbatim}}
		\end{minipage}
	\item \vspace{2ex} \begin{minipage}{10cm}{	\begin{verbatim}
X = bridget
Y = emily
						\end{verbatim}}
		\end{minipage}
	\item \vspace{2ex} \begin{minipage}{10cm}{	\begin{verbatim}
X = bridget
Y = donna  
						\end{verbatim}}
		\end{minipage}
	\item \vspace{2ex} \begin{minipage}{10cm}{	\begin{verbatim}
X = caroline
Y = emily
						\end{verbatim}}
		\end{minipage}
	\item \vspace{2ex} \begin{minipage}{10cm}{	\begin{verbatim}
X = anne
Y = bridget
						\end{verbatim}}
		\end{minipage}
	\item \vspace{2ex} \begin{minipage}{10cm}{	\begin{verbatim}
X = bridget
Y = caroline
						\end{verbatim}}
		\end{minipage}
	\item \vspace{2ex} \begin{minipage}{10cm}{	\begin{verbatim}
X = caroline
Y = donna
						\end{verbatim}}
		\end{minipage}
	\item \vspace{2ex} \begin{minipage}{10cm}{	\begin{verbatim}
X = donna
Y = emily
						\end{verbatim}}
		\end{minipage}
\end{enumerate}

\subsection*{Question 3.2}
\verb?Fatal Error: local stack overflow?.

\subsection*{Question 3.3}
\verb?Fatal Error: local stack overflow?.

\subsection*{Question 4.2}
6 �tapes puis:
\begin{verbatim}
true ? ;
Fatal Error: local stack overflow
\end{verbatim}

\subsection*{Question 4.3}
Nous obtenons 10 solutions:
\begin{enumerate}
	\item\vspace{1ex} \begin{minipage}{10cm}{	\begin{verbatim}
X = anne
Y = bridget  
						\end{verbatim}}
		\end{minipage}
	\item \vspace{2ex} \begin{minipage}{10cm}{	\begin{verbatim}
X = bridget
Y = caroline  
						\end{verbatim}}
		\end{minipage}
	\item \vspace{2ex} \begin{minipage}{10cm}{	\begin{verbatim}
X = caroline
Y = donna  
						\end{verbatim}}
		\end{minipage}
	\item \vspace{2ex} \begin{minipage}{10cm}{	\begin{verbatim}
X = donna
Y = emily  
						\end{verbatim}}
		\end{minipage}
	\item \vspace{2ex} \begin{minipage}{10cm}{	\begin{verbatim}
X = anne
Y = caroline  
						\end{verbatim}}
		\end{minipage}
	\item \vspace{2ex} \begin{minipage}{10cm}{	\begin{verbatim}
X = bridget
Y = donna  
						\end{verbatim}}
		\end{minipage}
	\item \vspace{2ex} \begin{minipage}{10cm}{	\begin{verbatim}
X = caroline
Y = emily  
						\end{verbatim}}
		\end{minipage}
	\item \vspace{2ex} \begin{minipage}{10cm}{	\begin{verbatim}
X = anne
Y = donna  
						\end{verbatim}}
		\end{minipage}
	\item \vspace{2ex} \begin{minipage}{10cm}{	\begin{verbatim}
X = bridget
Y = emily  
						\end{verbatim}}
		\end{minipage}
	\item \vspace{2ex} \begin{minipage}{10cm}{	\begin{verbatim}
X = anne
Y = emily  
						\end{verbatim}}
		\end{minipage}
\end{enumerate}
puis: \verb?Fatal Error: local stack overflow?.

\newpage
\part{Entiers de \bsc{Peano}}
\subsection*{Question 6.1}
\begin{verbatim}
entier(zero).
entier(succ(X)) :- entier(X).
\end{verbatim}

\subsection*{Question 6.2}
Il y a une infinit� de successeurs de \verb?zero?.

subsection*{Question 7.1}
\begin{verbatim}
inf_ou_egal(zero, X) :- entier(X).
inf_ou_egal(succ(X),succ(Y)) :-  inf_ou_egal(X,Y).
\end{verbatim}

subsection*{Question 7.2}
\begin{enumerate}
\item true
\item false
\item \begin{itemize}
	\item \verb?X = zero ;?
	\item \verb?X = succ(zero) ;?
	\item \verb?X = succ(succ(zero)) ;?
	\end{itemize}
\item \begin{itemize}
	\item \verb?X = succ(zero) ;?
	\item \verb?X = succ(succ(zero)) ;?
	\item \verb?X = succ(succ(succ(zero))) ;?
	\item \verb?X = succ(succ(succ(succ(zero)))) ;?
	\item \dots
	\end{itemize}
	Toutes les pr�dicats � partir de \verb?succ(zero)? satisfont le but \verb?inf_ou_egal(succ(zero),Z)?.
\end{enumerate}

\subsection*{Question 8.1}
\begin{verbatim}
add(zero, X, X) :- entier(X).
add(X, zero, X) :- entier(X).
add(succ(X),succ(Y),succ(succ(Z))) :- add(X,Y,Z).
\end{verbatim}

\subsection*{Question 8.2}
\begin{enumerate}
\item \verb?false?
\item \verb?true?
\item \verb?R= succ(succ(succ(zero)))?
\item \verb?Z= succ(succ(zero))?
\item \begin{itemize}
		\item\vspace{1ex} \begin{minipage}{10cm}{	\begin{verbatim}
X = zero
Y = succ(zero)  
						\end{verbatim}}
		\end{minipage}
		\item\vspace{2ex} \begin{minipage}{10cm}{	\begin{verbatim}
X = succ(zero)
Y = succ(succ(zero))  
						\end{verbatim}}
		\end{minipage}
		\item\vspace{1ex} \begin{minipage}{10cm}{	\begin{verbatim}
X = succ(succ(zero))
Y = succ(succ(succ(zero)))  
						\end{verbatim}}
		\vspace{1ex}
		\end{minipage}
		\item \dots
	\end{itemize}
\end{enumerate}

\subsection* {Question 8.3}
\begin{verbatim}
sub(X, zero, X) :- entier(X).
sub(succ(X),succ(Y),Z) :- sub(X,Y,Z).
\end{verbatim}

\subsection*{Question 9}
\begin{verbatim}
mult(zero,X,zero) :- entier(X).
mult(X,zero,zero) :- entier(X).
mult(succ(X),Y,T) :- mult(X,Y,Z), add(Y,Z,T).
mult(X,succ(Y),T) :- mult(X,Y,Z), add(X,Z,T).
\end{verbatim}
\end{document}