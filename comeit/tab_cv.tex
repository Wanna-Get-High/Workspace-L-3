\documentclass[a4paper,11pt]{article}
\usepackage{a4wide}
\usepackage{lscape}
\usepackage{longtable}
\usepackage[utf8]{inputenc}
\usepackage[francais]{babel}
\usepackage{palatino}
\usepackage{graphicx}
\usepackage{multicol}
\usepackage{eurosym}
\usepackage{ifthen}
\usepackage{url}
\title{CV}
\author{Antonia}
\begin{document}
\begin{landscape}
  \footnotesize   
  \begin{longtable}{|l|l|l|p{4cm}|p{4cm}|l|l|p{4cm}|p{4cm}|}
    \caption{Echange C.V. François Lepan}\\
    \hline \bfseries François&\bfseries envoi&\bfseries reçu&\bfseries
    commentaires positifs&\bfseries points à améliorer&\bfseries
    exp. c.&\bfseries c. reçu&\bfseries commentaires positifs&\bfseries points à
    améliorer\\ \hline \hline
    \endfirsthead
    \hline \bfseries Antonia&\bfseries envoi&\bfseries reçu&\bfseries
    commentaires positifs&\bfseries points à améliorer&\bfseries
    exp. c.&\bfseries c. reçu&\bfseries commentaires positifs&\bfseries points à
    améliorer\\ \hline \hline
    \endhead 
    \hline \multicolumn{9}{r}{\emph{Suite sur la page suivante}}
    \endfoot
    \hline
    \endlastfoot
    \hline
             
    Amara T.  & 24/10 & 2/11 %% incomplet
			  & \begin{itemize}
				\item informations importantes présentes
				\item plutôt sobre et donc lisible
				\item rubrique bien placé
				\item pas de photo :)
				\end{itemize}
			  & \begin{itemize}
				\item agencement  : dates  + tabulation + intituler du poste (en dessous et aligné à l'intituler les informations)
				\item évite de mettre trop de police différentes (une ou 2 suffit)
				\item réserver le gras pour les informations a faire sortir comme compétences info ou compétence d'activité profesionnelles
				\item "Utilisation de Word, Power Point, Paint" ... inutile t'es informaticienne pas secrétaire ou designer :)
				\item adresse mail non souligner et pas de lien web
				\end{itemize}
			  & 14/11 & ?
			  & \begin{itemize}
				\item ?
				\end{itemize}
			  & \begin{itemize}
				\item ?
				\end{itemize} \\ \hline

    Arthur A. & 24/10 & 16/11 %% incomplet
			  & \begin{itemize}
				\item bien agence on s'y retrouve facilement
				\item toutes les informations importantes sont presente
				\item pas de photo :)
				\end{itemize}
			  & \begin{itemize}
				\item police particuliere : sa pique aux yeux :)
				\item tu devrais rajouter une partie Competences pour Informatique et Langue
				\end{itemize}
			  & 29/11 & ?
			  & \begin{itemize}
				\item ?
				\end{itemize}
			  & \begin{itemize}
				\item ?
				\end{itemize} \\ \hline

    Benjamin M. & 24/10 & & & & & & & \\ \hline %% rien recu
    
    Benjamin V. & 24/10 & 13/10 %% incomplet
    			& \begin{itemize}
				  \item bien agencé, on arrive à lire facilement
				  \item complet, on y retrouve toutes les informations importantes
				  \item sobre
				  \end{itemize}
				& \begin{itemize}
				  \item les parties ne sont pas très visible, peut être les mètres plus en avant (souligné, mettre en gras,etc)
			      \end{itemize} 
    			& 14/11 & ?
    			& \begin{itemize}
				  \item ?
				  \end{itemize}
			    & \begin{itemize}
				  \item ?
				  \end{itemize} \\ \hline 

    Camille R. 	& 24/10 & 6/11
				& \begin{itemize}
				  \item les informations principales sont présentes
				  \item rubrique bien placé
				  \item pas de photo :)
				  \end{itemize}
				& \begin{itemize}
				  \item trop de couleur (2 grand max)
				  \item rubrique expérience compétences = imboufable : essaye de réduire et d'aligner : date + tabulation + info avec le même alignement entre les rubriques
				  \item évite de souligner met jus en gras
				  \item Langue et autres compétences --> Langue et compétences techniques
				  \item "Langage  connu" : java peut être ?
				  \end{itemize}
				& 14/11 & 6/11
				& \begin{itemize}
				  \item de bon centres d'intérêt
				  \item pas de fautes
				  \end{itemize}
				& \begin{itemize}
				  \item trop classique, banal
				  \item dissocié les projets scolaire des expériences professionnelles
				  \item manques de couleur
				  \end{itemize} \\ \hline

    Cesar S.  & 24/10 & 31/10 %% incomplet
			  & \begin{itemize}
				\item bien agencé on lit facilement
				\item les informations importantes y sont présentes
				\end{itemize}
			  & \begin{itemize}
				\item réserver le gras pour les informations a faire sortir comme compétences info ou compétence d'activité profesionnelles
				\item ajouter les compétences acquise au cours des expériences professionnelles comme (autonome, travail en équipe, etc)
				\end{itemize}
			  & 14/11 & ?
			  & \begin{itemize}
				\item ?
				\end{itemize}
			  & \begin{itemize}
				\item ?
				\end{itemize} \\ \hline

    Donovan W. 	& 24/10 & 19/10 %% incomplet
    			& \begin{itemize}
				  \item bien agencé on lit facilement
				  \item il est complet, toutes les infos utiles y sont
				  \item sobre mais efficace (pas trop de couleur).
				  \item pas de photo :)
				  \end{itemize}
				& \begin{itemize}
				  \item compétences bureautique : "open office, Microsoft office" ... à enlever tout informaticien sait utiliser un éditeur de texte
				  \item expérience professionnelle : je l'aurai mis au dessus de compétences informatiques et met en évidences les éléments clef (autonomie, travaille en équipe, etc) de ton expérience
				  \item réserver le gras pour les informations a faire ressortir pour un poste donné
				  \item date de naissance facultatif si tu lui donne ton age
				  \end{itemize} 
    			& 14/11 & ?
    			& \begin{itemize}
				  \item ?
				  \end{itemize}
			    & \begin{itemize}
				  \item ?
				  \end{itemize} \\ \hline 

    Antonia L. 	& 24/10 & 24/10 
				& \begin{itemize}
				  \item bien agencé pour formation et activite professionelle
				  \item sobre mais efficace
				  \item pas de photo :)
				  \end{itemize}
				& \begin{itemize}
				  \item compétences : ajoute BDD peut être (mysql, etc) aligne avec formation et act prof
				  \item évite l'italique à tout bout de champ
				  \item centre d'intérêt : police trop petite ça veut dire  : "c'est la mais sans importance" alors que le recruteur c'est surement là ou il va le plus s’arrêter sur ton CV
				  \item pas de capitale d'imprimerie pour le nom de famille
				  \end{itemize}
				& 14/11 & 28/11
				& \begin{itemize}
				  \item cv aéré et bien détaillé
				  \item informations personnelles bien mises en avant
				  \end{itemize}
				& \begin{itemize}
				  \item pour les catégories les mettres plutôt en gras au lieu de souligner
				  \item pareil pour les intitulés de diplôme et toutes les choses à mettre en avant, essayer de jouer avec la police (car ce serait difficile de tout lire en 10s)
				  \item il n' y a rien sur ta formation actuelle
				  \end{itemize} \\ \hline

    Gauvain M. 	& 24/10 & 23/10 
				& \begin{itemize}
				  \item sobre mais efficace
				  \item bien agencé on lit facilement
				  \item pas de photo :)
				  \end{itemize}
				& \begin{itemize}
				  \item permis B à mettre en dessous de ton adresse mail.
				  \item réserver le gras pour les informations a faire sortir comme compétences info ou compétence d'activité profesionnelles
				  \item ajouter compétences pour chaque activités professionnelles comme (autonomie, travail en équipe,etc)
				  \end{itemize}
				& 14/11 & 29/10s
				& \begin{itemize}
				  \item catégories dans le bon ordre
				  \item catégories bien mis en avant
				  \item bonne mise en avant des info personnelles
				  \end{itemize} 
				& \begin{itemize}
				  \item prob. de typographie, enlever les soulignés
				  \item où est la formation actuelle ?
				  \item date du stage à mettre dans la date et pas dans la description
				  \end{itemize} \\ \hline

    Jérémie S.  & 24/10 & 13/11 %% incomplet
				& \begin{itemize}
				  \item bien agencé
				  \item catégorie dans l'ordre
				  \item sobre
				  \end{itemize}
				& \begin{itemize}
				  \item "Anglais Bilingue" --> anglais courant
				  \item "Loisir Echec" --> loisirs jouer aux échec (sinon sa fait culture de l’échec je trouve :)
				  \item essaye d'aérer Expériences Professionnelle on a du mal a s'y retrouver dedans
				  \end{itemize} 
				& 14/11 & ?
				& \begin{itemize}
				  \item ?
				  \end{itemize}
				& \begin{itemize}
				  \item ?
				  \end{itemize} \\ \hline

    Lucie B. & 24/10 & & & & & & & \\ \hline %% rien recu

    Lucille D.  & 24/10 & 19/10
				& \begin{itemize}
				  \item toutes les informations importantes y sont
				  \item sobre mais efficace
				  \item compétences mis en avant pour chaque expérience
				  \end{itemize}
				& \begin{itemize}
				  \item aligne toutes les informations au même niveau sa évite au recuteur de zigzaguer et jeter ton CV :)
				  \item dans Formation : évite -> Connaissances acquises  pas besoin si tu liste des compétences il comprendra.
				  \end{itemize}
				& 14/11 & 13/11
				& \begin{itemize}
				  \item il est complet, toutes les infos utiles y sont
				  \item lisible, on distingue bien toutes les catégories
				  \end{itemize}
				& \begin{itemize}
				  \item présentation pas très funky, il manque peut-être d'effets genre de l'italique ou des tailles de polices différentes pour le rendre un peu moins classique
				  \item il faudrait préciser tes dates de stage et de projets
				  \end{itemize} \\ \hline

    Mengmeng T. & 24/10 & 2/11 %% incomplet
				& \begin{itemize}
				  \item informations importantes présentes
				  \item rubrique bien marqué
				  \item pas de photo :)
				  \end{itemize}
				& \begin{itemize}
				  \item expériences -> expériences professionnelles
				  \item autres activités --> réalisations
				  \item CENTRES D'INTERET --> CENTRES D’INTÉRÊT
				  \item aligne tout comme pour Formation pour une meilleur lecture.
				  \item réserver le gras pour les informations a faire sortir comme compétences info ou compétence d'activité profesionnelles
				  \end{itemize} 
				& 14/11 & ?
				& \begin{itemize}
				  \item ?
				  \end{itemize}
				& \begin{itemize}
				  \item ?
				  \end{itemize} \\ \hline

    Nadia N.  & 24/10 & 13/11  %% incomplet
			  & \begin{itemize}
				\item sobre.
				\item les informations générales sont présentes.
				\end{itemize}
			  & \begin{itemize}
				\item éviter les pastilles en début d'information : agencement -> dates + tabulation + intituler du poste
				\item réserver le gras pour les informations a faire sortir comme compétences info ou compétence d'activité profesionnelles
				\item ajouter les compétences acquise au cours des expériences professionnelles comme (autonome, travail en équipe, etc)
				\item 'compétences particulières' changer en 'Langue' et met permis B en dessous du courriel
				\end{itemize}
			  & 14/11 & ?
			  & \begin{itemize}
				\item ?
				\end{itemize}
			  & \begin{itemize}
				\item ?
				\end{itemize} \\ \hline

    Omar C. & 24/10 & 5/11 %% incomplet
			& \begin{itemize}
			  \item ya de l'information mais incomplet
			  \item sobre pas tp de couleur
			  \item pas de photo :)
			  \end{itemize}
			& \begin{itemize}
			  \item Loisirs --> centres d'intérêt.
			  \item Connaissances informatiques --> Compétences techniques
			  \item email : évite hotmail sa fait pas serieux -> omar.chahbouni@ gmail ou autre
			  \item formation : rassemble tes 3 années de licence et si tu veux gagner de la place écrit en dessous les compétences non informatique acquise durant celle-ci
			  \item agencement  : dates  + tabulation + intituler (en dessous et aligné à l'intituler les informations) pour formation
			  \item "Manipulation aisée des logiciels WORD, EXCELL, POWER POINT" ... inutile t'es informaticien pas secrétaire
			  \item "Maitrise de plusieurs langages de programmation (Java,C,etc…)" ... développe affiche tout ce que tu sais faire regarde les CV envoyer ya plein d'exemple
			  \item Experience professionnelle ?? sinon Réalisation et t'y met ce que tu a fait comme projet en Licence
			  \end{itemize}
			& 14/11 & ?
			& \begin{itemize}
			  \item ?
			  \end{itemize}
			& \begin{itemize}
			  \item ?
			  \end{itemize} \\ \hline

    Ouardia M.  & 24/10 & 30/10 %% incomplet
				& \begin{itemize}
				  \item les informations importantes y sont présentes
				  \item sobre mais efficace
				  \item pas de photo :)
				  \end{itemize}
				& \begin{itemize}
				  \item évites les tableaux sa rajoute du surplus d'informations pour les yeux et donc plus difficile de trouvé l'information importante
				  \item réserver le gras pour les informations a faire sortir comme compétences info ou compétence d'activité profesionnelles
				  \end{itemize}
				&  14/11 & ? 
				& \begin{itemize}
				  \item ?
				  \end{itemize}
				& \begin{itemize}
				  \item ?
				  \end{itemize} \\ \hline

    Quentin F. & 24/10 & & & & & & & \\ \hline %% rien recu

    Rabab B.  & 24/10 & 5/11 %% incomplet
			  & \begin{itemize}
				\item bien agencé (mais aligne tout au même niveau)
				\item clair et lisible.
				\item pas de photo :)
				\end{itemize}
	  	      & \begin{itemize}
				\item réserver le gras pour les informations a faire sortir comme compétences info ou compétence d'activité profesionnelles
				\item réalisation ajoute les compétence mis en oeuvre pour cette réalisation come (travaille en équipe, autonomie, etc)
				\item "Bureautique  OpenOffice/LibreOffice, Word, Excel " ... inutile t'es informaticien pas secrétaire
				\end{itemize}
			  & 14/11 & ?
			  & \begin{itemize}
				\item ?
				\end{itemize}
			  & \begin{itemize}
				\item ?
				\end{itemize} \\ \hline

    Sebastien L. & 24/10 & 31/10 %% incomplet
				 & \begin{itemize}
				   \item bien agence on s'y retrouve facilement
				   \item toutes les informations importantes sont presentes
				   \item pas de photo :)
			 	   \end{itemize}
				 & \begin{itemize}
				   \item nom de famille pas en capitale d'imprimerie
				   \item tu devrais tabuler les informations des differentes categories au meme niveau
				   \end{itemize} 
				 & 29/11 & 
				 & \begin{itemize}
				   \item ?
				   \end{itemize}
				 & \begin{itemize}
				   \item ?
				   \end{itemize} \\ \hline

    Sebastien R.  & 24/10 & 31/10 %% incomplet
				  & \begin{itemize}
					\item informations importantes présentes
					\item pas de photos :)
					\end{itemize}
				  & \begin{itemize}
					\item diplômes : met 2008 - 2011 licence informatique sinon sur le coup sa fait un trou de 3 ans ou t'as rien fait
					\item diplômes --> formations
					\item atouts --> compétences
					\item ajoute une rubrique centres d'intérêt (sans s a intérêt) parce que le recruteur  va galérer a les trouvé sinon.
					\item expériences professionnelles mal agencé: fait le à l'image de ta rubrique diplômes et met en dessous et aligné de l'intituler du poste les compétences acquise et non aligne a gauche en dessous de la date.
					\item "maîtrise du pack office" ... t'es informaticien pas secrétaire, le recruteur ne va pas baser son choix sur cette compétence ci.
					\end{itemize}
				  & 14/11 & ?
				  & \begin{itemize}
					\item ?
					\end{itemize}
				  & \begin{itemize}
					\item ?
					\end{itemize} \\ \hline

    Stéfan D. & 24/10 & 19/10
			  & \begin{itemize}
				\item bien agencé on trouve l'information facilement
				\item toutes les info sont présentes
				\item sobre mais efficace
				\item pas de photo :)
				\end{itemize}
			  & \begin{itemize}
				\item expérience professionnelle je l'aurai mis au dessus de compétences informatiques
				\item réserver le gras pour les informations a faire sortir
				\item date de naissance facultatif (il connait ton age, c'est le prendre pour un c** de lui donnée ta date : "j'ai 21 ans je suis né en 1990")
				\item rubrique : divers = "poubelle", ça veut dire : "j'ai fais ça mais je savais pas ou le placer pour te faire plaisir". -> expérience personnelle
				\item rubrique : projet -> rajoute expérience professionnelle devant pour bien voir que c'en est une autre plutôt qu'un projet aléatoire
				\item centres d’intérêts ----> centres d’intérêt
				\end{itemize}
			  & 14/11 & 1/11
			  & \begin{itemize}
				\item bon agencement des catégories
				\item informations biens mises en avant
				\end{itemize}
			  & \begin{itemize}
				\item peut-être supprimer les espaces au début de ton CV et mettre ton nom et ton prénom plus en avant.
				\item changer de police, le Times New Roman est presque plus utilisée de nos jours, tente la police Tahoma par exemple.
				\item enlever les "deux points" et les sous-lignés, mets en avant les infos principales en mettant en gras ou en italique.
				\item rajouter la formation actuelle.
				\end{itemize} \\ \hline

    Sylvain D.  & 24/10 & 23/10
    			& \begin{itemize}
				  \item informations importantes présentes
  				  \item sobre mais efficace
				  \end{itemize}
				& \begin{itemize}
				  \item brevet ... à proscrire du CV inutile au possible. si tu veux gagner de la place liste les compétences acquise au cour des formations suivis autre que informatique.
				  \item réserver le gras pour les informations a faire sortir comme compétences info ou compétence d'activité profesionnelles
				  \item éviter les tirets au début de chaque sous-rubrique met a gauche l'intituler et/ou la date et a droite l'information (avec toutes les informations aligné pour une meilleur lecture)
				  \end{itemize} 
    			& 14/11 & 26/11
    			& \begin{itemize}
				  \item cv très complet qui contient toutes les informations importantes
				  \item clair et bien organisé
				  \item simple et efficace
				  \end{itemize}
				& \begin{itemize}
				  \item manque de couleur
				  \item police peut être trop classique
				  \item expérience professionnelle au singulier (d'après le prof de com)
				  \end{itemize} \\ \hline
  \end{longtable}
\end{landscape}
\end{document}
